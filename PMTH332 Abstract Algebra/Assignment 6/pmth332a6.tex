\documentclass{article}

\usepackage{amsmath}
\usepackage{amssymb}
\usepackage{listings}
\usepackage{color}
\usepackage{hhline}
\usepackage{geometry}
\geometry{
    a4paper,
    left=25mm,
    top=25mm
}

\definecolor{codegreen}{rgb}{0,0.6,0}
\definecolor{codegray}{rgb}{0.5,0.5,0.5}
\definecolor{codepurple}{rgb}{0.58,0,0.82}
\definecolor{backcolour}{rgb}{0.95,0.95,0.92}
    
\lstdefinestyle{mystyle}{
    backgroundcolor=\color{backcolour},   
    commentstyle=\color{codegreen},
    keywordstyle=\color{magenta},
    numberstyle=\tiny\color{codegray},
    stringstyle=\color{codepurple},
    basicstyle=\footnotesize,
    breakatwhitespace=false,         
    breaklines=true,                 
    captionpos=b,                    
    keepspaces=true,                 
    numbers=left,                    
    numbersep=5pt,                  
    showspaces=false,                
    showstringspaces=false,
    showtabs=false,
    tabsize=4
}
\lstset{style=mystyle}

\newcommand\tab[1][1cm]{\hspace*{#1}}
\setlength{\parindent}{0pt}

\iffalse <Subject> Assignment <Assignment number> \fi
\title{PMTH332 Assignment 6}
\iffalse <dd> <Month> <yyyy> \fi
\date{5 October 2018}
\author{Jayden Turner (SN 220188234)}

\begin{document}
\maketitle
\pagenumbering{arabic}

\section*{Question 1}

\hfill \break
Take $n \in \mathbb{Z}$. If $3 \mid n$, then $n^{33} \equiv n\mod3
\implies 3 | n^{33} - n$. If $3 \nmid n$, then

\begin{align}
    n^{33} - n = n(n^{32} - 1) = n((n^2)^{16} - n)
        = n((n^{\phi(3)})^{16} - n) &\equiv n(1 - 1)\mod3 \label{eq:1-1}\\
    &\equiv 0\mod3\nonumber
\end{align}

where $\phi(n)$ is the Euler phi function, and (\ref{eq:1-1}) holds
by Euler's theorem. Thus $3 | n^{33} - n$ for all integer $n$.
Similiarly, if $5 \mid n$, then $n^{33} \equiv n\mod5 \implies 5 | n^{33} - n$.
Otherwise,

\begin{align}
    n^{33} - n = n(n^{32} - 1) = n((n^4)^8 - 1)
        = n((n^{\phi(n)})^8 - 1) &\equiv n(1 - 1)\mod5 \label{eq:1-2}\\
    &\equiv 0\mod5\nonumber
\end{align}

where (\ref{eq:1-2}) holds by Euler's theorem, and hence $5 | n^{33} - n$ for
all integer $n$. Therefore, as both 3 and 5 divide $n^{33} - n$, and $\gcd(3, 5) = 1$,
it must hold that $3\cdot5 = 15 | n^{33} - n$ as required.

\section*{Question 2}

Given a field $F$, the ring of polynomials over $F$, $F[x]$ is an integral
domain. define $d: F[x] \to \mathbb{N}$ as

\begin{equation}
    d(\alpha) := \begin{cases}
        0, & \alpha = 0\\
        2^{\text{deg}(\alpha)}, & \alpha \neq 0
    \end{cases}
\end{equation}

By definition, $d(\alpha) = 0$ if and only if $\alpha = 0$. Further,
$d(1) = 2^0 = 1$, and as $\text{deg}(\alpha) \geq 0$,
$d(\alpha\beta) = 2^{\text{deg}(\alpha)}2^{\text{deg}(\beta)}
    \geq 2^{\text{deg}(\alpha)} = d(\alpha)$, given $\beta \neq 0$. Thus
$d$ satisfies the first two axioms of a Euclidean function. To see that it
satisfies the third, take $\alpha, \beta \in F[x]$ such that
$n = \text{deg}(\alpha) \geq \text{deg}(\beta) = m$. Then define $q, r \in F[x]$ as

\begin{equation*}
    q := \frac{a_n}{b_m}x^{n - m}
\end{equation*}

and

\begin{equation*}
    r := \alpha - q\beta
\end{equation*}

Clearly $\alpha = q\beta + r$. Expanding $r$ we see that

\begin{align*}
    r &= a_nx^n + a_{n - 1}x^{n - 1} + ... + a_0 - \frac{a_n}{b_m}b_mx^{n}
        - ... - \frac{a_n}{b_m}b_0x^{n - m}\\
    &= \left(a_{n - 1} - \frac{a_n}{b_m}b_{m - 1}\right)x^{n - 1} + ...
        + \left(a_{n - m} - \frac{a_n}{b_m}b_0\right)x^{n - m} + ...
\end{align*}

which shows that $\text{deg}(r) \leq n - 1 < \text{deg}(\beta)$, and hence
$d$ is a Euclidean function. Thus $F[x]$ is a Euclidean domain, which is
a principal ideal domain, by Theorem 17.3. Therefore, given any ideal $I$
in $F[x]$, $I = (\alpha)$ for some $\alpha in F[x]$. If $I$ is a prime
ideal, then $\alpha$ is prime in $F[x]$. But by Theorem 17.14, if $\alpha$
is prime, then $(\alpha) = I$ is maximal. Therefore, all proper prime ideals
of $F[x]$ are maximal.

\section*{Question 3}
\section*{Question 4}
\section*{Question 5}

\end{document}
