\documentclass{article}

\usepackage{amsmath}
\usepackage{amssymb}
\usepackage{listings}
\usepackage{color}
\usepackage{hhline}
\usepackage{geometry}
\geometry{
    a4paper,
    left=25mm,
    top=25mm
}

\definecolor{codegreen}{rgb}{0,0.6,0}
\definecolor{codegray}{rgb}{0.5,0.5,0.5}
\definecolor{codepurple}{rgb}{0.58,0,0.82}
\definecolor{backcolour}{rgb}{0.95,0.95,0.92}
    
\lstdefinestyle{mystyle}{
    backgroundcolor=\color{backcolour},   
    commentstyle=\color{codegreen},
    keywordstyle=\color{magenta},
    numberstyle=\tiny\color{codegray},
    stringstyle=\color{codepurple},
    basicstyle=\footnotesize,
    breakatwhitespace=false,         
    breaklines=true,                 
    captionpos=b,                    
    keepspaces=true,                 
    numbers=left,                    
    numbersep=5pt,                  
    showspaces=false,                
    showstringspaces=false,
    showtabs=false,
    tabsize=4
}
\lstset{style=mystyle}
\setlength{\parindent}{0pt}

\newcommand\cart[2]{{#1}\times{#2}}

\iffalse <Subject> Assignment <Assignment number> \fi
\title{PMTH332 Assignment 6}
\iffalse <dd> <Month> <yyyy> \fi
\date{5 October 2018}
\author{Jayden Turner (SN 220188234)}

\begin{document}
\maketitle
\pagenumbering{arabic}

\section*{Question 1}

If $\phi$ is the Euler phi function, then we have $\phi(3) = 2$ and
$\phi(5) = 4$. This, combined with Euler's theorem gives

\begin{align}
    a^2 &\equiv 1\mod3 \label{eq:1-1}\\
    b^4 &\equiv 1\mod5 \label{eq:1-2}
\end{align}

where $a$ and $b$ are integers coprime to 3 and 5 respectively.

\hfill \break
Take $n \in \mathbb{Z}$. If $3 \mid n$, then
$n^{33} \equiv n\mod3 \implies n^{33} - n \equiv 0\mod3$. Otherwise,

\begin{align*}
    n^{33} - n = n(n^{32} - 1) &= n((n^2)^{16} - 1)\\
    (\ref{eq:1-1}) \implies &\equiv n(1 - 1)\mod3\\
    &\equiv 0 \mod 3
\end{align*}

That is, for all $n \in \mathbb{Z}$, $3 | n^{33} - n$. Likewise, if
$5 | n$, then $n^{33} \equiv n\mod5 \implies n^{33} - n \equiv 0\mod5$.
Otherwise,

\begin{align*}
    n^{33} - n = n(n^{32} - 1) &= n((n^4)^8 - 1)\\
    (\ref{eq:1-2}) \implies &\equiv n(1 - 1)\mod5\\
    &\equiv 0 \mod 5
\end{align*}

That is, $5 | n^{33} - n$ for all $n \in \mathbb{Z}$. Therefore, as both
3 and 5 divide $n^{33} - n$, and $\gcd(3, 5) = 1$, it must hold that
$3\cdot5 = 15 | n^{33} - n$.

\section*{Question 2}

Let $F$ be a field. Then $F[x]$ is an integral domain permitting Euclidean
function $d: F[x] \to \mathbb{N}$, where

\begin{align*}
    d := \begin{cases}
        0, & \alpha = 0\\
        2^{\deg{\alpha}}, & \alpha \neq 0
    \end{cases}
\end{align*}

To see that this is a Euclidean function, we verify that $d$ satisfies
the three required axioms. Observe that E1 holds by definition, and that
$d(1) = 2^{\deg 1} = 2^0 = 1$. Further, if $\alpha, \beta \in F[x]$ and
$\beta \neq 0$, then
$d(\alpha\beta) = 2^{\deg(\alpha\beta)}
    = 2^{\deg\alpha}2^{\deg\beta}
    \geq 2^{deg\alpha} = d(\alpha)$
so E2 also holds. Finally, given arbitrary $\alpha, \beta in F[x]$ with
expansions $\alpha = \sum_{i = 0}^n a_ix^i$, $\beta = \sum_{j = 0}^m b_jx^j$,
where $n \geq m$, it is possible to define

\begin{equation*}
    q := \frac{a_n}{b_m}x^{n - m}
\end{equation*}

and

\begin{equation} \label{eq:2-1}
    r := \alpha - q\beta
\end{equation}

such that $\alpha = q\beta + r$. Expanding (\ref{eq:2-1}), we get

\begin{align*}
    \alpha - q\beta &= \sum_{i = 0}^n a_ix^i - \frac{a_n}{b_m}x^{n - m}\sum_{i = 0}^m b_ix^i\\
    &= a_nx^n + \sum_{i = 0}^{n - 1}a_ix^i - a_nx^{n} - \sum_{i = n - m}^{n - 1}
        \frac{a_n}{b_m}b_{i - n + m}x^i\\
    &= \sum_{i = 0}^{n - 1}c_ix^i
\end{align*}

with

\begin{equation*}
    c_i := \begin{cases}
        a_i, & 0 \leq i \leq n - m - 1\\
        \frac{a_n}{b_m}b_i, & n - m \leq i \leq n - 1
    \end{cases}
\end{equation*}

Hence it is always possible to find $q, r$ such that $\alpha = q\beta + r$ with
$d(r) < d(\beta)$, and axiom E3 is satisfied. $d$ is therefore a Euclidean function on
integral domain $F[x]$, making $F[x]$ a Euclidean domain, and hence a principal
ideal domain, by Theorem 17.3.

\hfill\break
Take arbitrary prime ideal $P$ of $F[x]$. As $F[x]$ is a pid, $P = (\alpha)$ for some
$\alpha \in F[x]$, making $\alpha$ prime in $F[x]$ by definition. By
Theorem 17.14, $(\alpha) = P$ must be maximal. Hence every prime ideal of $F[x]$,
for field $F$, is maximal, as required.

\section*{Question 3}

As shown in Question 2, if $F$ is a field then $F[x]$ is a principal ideal domain.
Therefore, as $\gcd(f(x), g(x)) = 1$, there exist $u(x), v(x) \in F[x]$ satisfying

\begin{equation} \label{eq:3-1}
    1 = u(x)f(x) + v(x)g(x)
\end{equation}

Given that $f(x)\mid h(x)$ and $g(x)\mid h(x)$, we have

\begin{align}
    h(x) &= s(x)f(x) \label{eq:3-2}\\
    h(x) &= t(x)g(x) \label{eq:3-3}
\end{align}

for some $s(x), t(x) \in F[x]$. Multiply (\ref{eq:3-1}) by $h(x)$ to get

\begin{equation}
    h(x) = u(x)h(x)f(x) + v(x)h(x)g(x) \label{eq:3-4}
\end{equation}

Expanding $h(x)$ in the left summand with (\ref{eq:3-3}) and using
(\ref{eq:3-2}) in the right, we obtain

\begin{align*}
    h(x) &= u(x)t(x)f(x)g(x) + v(x)s(x)f(x)g(x)\\
    &= f(x)g(x)(u(x)t(x) + v(x)s(x))
\end{align*}

and hence $f(x)g(x)\mid h(x)$ as required.

\section*{Question 4}

\textbf{a)} As $\mathbb{Z}$ is principal ideal domain, so is $\mathbb{Z}_{12}$. That is,
every ideal $I$ is of the form $I = (a)$ for some $a \in \mathbb{Z}_{12}$. By Theorem 16.11,
every non-zero maximal ideal of a commutative unital ring is prime. Therefore, it is
possible to find the prime ideals of $\mathbb{Z}_{12}$ by finding the ideals generated
by each element of $\mathbb{Z}_{12}$ and determining which are maximal. The ideals of
$\mathbb{Z}_{12}$, excluding $(0)$, are

\begin{align*}
    (1) &= \mathbb{Z}_{12}\\
    (2) &= \{0, 2, 4, 6, 8, 10\}\\
    (3) &= \{0, 3, 6, 9\}\\
    (4) &= \{0, 4, 8\}\\
    (5) &= \mathbb{Z}_{12}\\
    (6) &= \{0, 6\}\\
    (7) &= \mathbb{Z}_{12}\\
    (8) &= \{0, 4, 8\}\\
    (9) &= \{0, 3, 6, 9\}\\
    (10) &= \{0, 2, 4, 6, 8, 10\}\\
    (11) &= \mathbb{Z}_{12}
\end{align*}

As prime ideals are, by definition, proper, we therefore have that the prime
ideals of $\mathbb{Z}_{12}$ are $(2)$ and $(3)$.

\hfill\break
\textbf{b)} $I = \cart{\mathbb{Z}}{\{0\}}$ is a prime ideal of $\cart{\mathbb{Z}}{\mathbb{Z}}$.
To see that it is an ideal, take $(a, 0) \in I$ and $(b, c) \in \cart{\mathbb{Z}}{\mathbb{Z}}$.
Then $(a, 0)\cdot(b, c) = (ab, 0) \in I$. To see that it is prime, take
$x = (a, b), y = (c, d) \in \cart{\mathbb{Z}}{\mathbb{Z}}$. If
$xy = (ac, bd)$ is in $I$, then either $c = 0$ or $d = 0$. Therefore
$xy \in I$ implies $x \in I$ or $y \in I$, which makes $I$ prime by definition. However,
$I$ is not maximal as $\cart{\mathbb{Z}}{p\mathbb{Z}}$ is prime for prime integer $p$, and
$I \subset \cart{\mathbb{Z}}{p\mathbb{Z}}$.

\section*{Question 5}

Evaluating $p(x) = x^3 + 2x + 3$ for each element of $\mathbb{Z}_5$ we get

\begin{align*}
    p(0) &\equiv 3\\
    p(1) &= 6 \equiv 1\mod5\\
    p(2) &= 15 \equiv 0\mod5\\
    p(3) &= 36 \equiv 1\mod5\\
    p(4) &= 75 \equiv 0\mod5
\end{align*}

Hence $p(x)$ has roots in $\mathbb{Z}_5$ at $x = 2, 4$. By the factor theorem,

\begin{equation*}
    p(x) = (x - 2)(x - 4)\beta \equiv (x + 3)(x + 1)\beta
\end{equation*}

where $\beta$ is an element of $\mathbb{Z}_5[x]$ of degree 1. Suppose
$\beta = ax + b$. Then,

\begin{align*}
    p(x) = x^3 + 2x + 3 &= (ax + b)(x + 3)(x + 1)\\
    &= (ax + b)(x^2 + 4x + 3)\\
    &= ax^3 + (4a + b)x^2 + (3a + 4b)x + 3b
\end{align*}

Equating coefficients, we get $a = 1$ and $4 + b = 0 \implies b \equiv 1\mod5$. Therefore
$p(x)$ can be decomposed into the product of irreducible polynomials

\begin{equation*}
    p(x) = (x + 1)^2(x + 3)
\end{equation*}

\end{document}
