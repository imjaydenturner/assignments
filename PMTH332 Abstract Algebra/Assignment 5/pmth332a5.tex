\documentclass{article}

\usepackage{amsmath}
\usepackage{amssymb}
\usepackage{listings}
\usepackage{color}
\usepackage{hhline}
\usepackage{geometry}
\geometry{
    a4paper,
    left=25mm,
    top=25mm
}

\definecolor{codegreen}{rgb}{0,0.6,0}
\definecolor{codegray}{rgb}{0.5,0.5,0.5}
\definecolor{codepurple}{rgb}{0.58,0,0.82}
\definecolor{backcolour}{rgb}{0.95,0.95,0.92}
    
\lstdefinestyle{mystyle}{
    backgroundcolor=\color{backcolour},   
    commentstyle=\color{codegreen},
    keywordstyle=\color{magenta},
    numberstyle=\tiny\color{codegray},
    stringstyle=\color{codepurple},
    basicstyle=\footnotesize,
    breakatwhitespace=false,         
    breaklines=true,                 
    captionpos=b,                    
    keepspaces=true,                 
    numbers=left,                    
    numbersep=5pt,                  
    showspaces=false,                
    showstringspaces=false,
    showtabs=false,
    tabsize=4
}
\lstset{style=mystyle}

\newcommand\tab[1][1cm]{\hspace*{#1}}
\setlength{\parindent}{0pt}

\iffalse <Subject> Assignment <Assignment number> \fi
\title{PMTH332 Assignment 5}
\iffalse <dd> <Month> <yyyy> \fi
\date{22 September 2018}
\author{Jayden Turner (SN 220188234)}

\begin{document}
\maketitle
\pagenumbering{arabic}

\section*{Question 1}

Consider the two non-zero matrices $A, B \in M(2; R)$ defined by

\begin{align*}
    A = \begin{pmatrix}
        a & 0\\
        0 & 0
    \end{pmatrix}
    &&
    B = \begin{pmatrix}
        0 & 0\\
        0 & a
    \end{pmatrix}
\end{align*}

then $AB = 0$, so $M(2; R)$ has zero divisors. Consider matrix $C$ given by

\begin{align*}
    C = \begin{pmatrix}
        0 & a\\
        0 & 0
    \end{pmatrix}
\end{align*}

then $AC = \begin{pmatrix}0 & a^2\\0 & 0\end{pmatrix}$ and $CA = 0$,
so $M(2; R)$ is not commutative.

\section*{Question 2}

Let $F$ be a finite integral domain and take non-zero $a \in F$. Define $F^*$
to be the set of non-zero elements of $F$, and define the map $\phi: F^* \to F^*$
by $\phi(x) = ax$.

\hfill\break
Suppose that for $x, y \in F^*$, $\phi(x) = \phi(y)$. Then

\begin{equation*}
    ax = ay \iff ax - ay = 0 \iff a(x - y) = 0
\end{equation*}

As $F$ is an integral domain, it has no zero-divisors. Therefore the above
implies that either $a = 0$ or $x - y = 0$. As $a$ is non-zero by choice,
it must hold that $x = y$. Therefore $\phi$ is injective. Further, as $F^*$
is finite, $\phi$ must be surjective. Therefore, as $1 \in F^*$, there
exists $x \in F^*$ so that $\phi(x) = ax = 1$. Hence, every non-zero element
of $F$ has multiplicative inverse, so $F$ is a field.

\section*{Question 3}

Let $R$ be a non-trivial integral domain with 1. The characteristic of $R$
is the integer $n$ such that $n\mathbb{Z} = \text{ker}\epsilon$, where
$\epsilon: \mathbb{Z} \to R$ is the unique homomorphism of unital rings, given
by $\epsilon(n) = n\cdot1_R$.

\hfill\break
Given that $n\mathbb{Z} = \{x \in \mathbb{Z} | x = nm, m \in \mathbb{Z}\}$ and
$\ker\epsilon = \{x \in \mathbb{Z} | x\cdot1_R = 0_R\}$, we can deduce that
if $n$ is the characteristic of $R$, then $nm\cdot1_R = (n\cdot1_R)(m\cdot1_R) = 0_R$
for all $m \in \mathbb{Z}$. $n = 0$ satisfies this, in which case $\mathbb{Z}$ is a
subring of $R$.

\hfill\break
Suppose $n \neq 0$ and $n$ is not prime. Then $n = kp$ for some prime $p < n$ and
natural number $k$. $n$ being the characteristic of $R$, we must have that for
non-zero $a \in R$, $na\cdot1_R = (k\cdot1_R)(p\cdot1_R)a = 0_R$. As $R$ is an
integral domain, $R$ has no zero-divisors. Hence it must hold that either $k = 0$
or $p = 0$. This is a contradiction of the assumption that $n$ is not prime, hence
$n$ must be so.

\hfill\break
Therefore, the characteristic of an integral domain with 1 must be either 0 or prime.

\section*{Question 4}

Given an integral domain $D$, consider
$\widetilde{D} = \{(b, a) | a, b \in D, a \neq 0\}$, and the equivalence
relation $(b, a) \sim (d, c) \iff bc = ad$. Let $F$ be the set of equivalence
classes of this equivalence relation. Define addition and multiplication on
$F$ as

\begin{align*}
    + : F \times F \to F, && ([(b, a)], [(d, c)]) \mapsto [(bc + ad, ac)]\\
    \times: F \times F \to F, && ([(b, a)], [(d, c)]) \mapsto [(bd, ac)]
\end{align*}

Firstly, we show that the given operations are well defined. Given
$(b, a), (b', a'), (d, c), (d', c') \in \widetilde{D}$,
where $(b, a) \sim (b', a')$ and $(d, c) \sim (d', c')$
we have that

\begin{align*}
    (b, a) + (d, c) &= (ad + bc, ac)\\
    &\text{and}\\
    (b', a') + (d', c') &= (a'd' + b'c', a'c')
\end{align*}

Given that $a'b = b'a$ and $c'd = d'c$, multiply the first by $cc'$,
the second by $aa'$, and add to get 

\begin{align*}
    cc'a'b + aa'c'd &= cc'b'a + aa'd'c\\
    bca'c' + ada'c' &= acb'c' + aca'd'\\
    a'c'(ad + bc) &= ac(a'd' + b'c')
\end{align*}

which shows that $(b, a) + (d, c) \sim (b', a') + (d', c')$. Concerning
multiplication, we have that

\begin{align*}
    (b, a) \times (d, c) &= (bd, ac)\\
    &\text{and}\\
    (b', a') \times (d', c') &= (b'd', a'c')
\end{align*}

Given that $a'b = b'a$ and $c'd = d'c$, multiply the first be the second
to get

\begin{align*}
    a'bc'd &= b'ad'c\\
    a'c'bd &= acb'd'
\end{align*}

Which shows that $(b, a) \times (d, c) \sim (b', a') \times (d', c')$. Hence both
binary operations are well-defined.

\hfill\break
We now proceed to prove that $F$ is a field. Firstly, as $D$ is an integral domain,
addition and multiplication under $F$ inherit commutativity. The additive identity
is $[(0, 1)]$, as

\begin{equation*}
    [(0, 1)] + [(b, a)] = [(a\cdot b + 0\cdot a, 1\cdot a)] = [(b, a)]
        = [(b, a)] + [(0, 1)]
\end{equation*}

for $[(b, a)] \in F$. Also, each element of $F$ has additive inverse
$[(-\frac{b}{a^2}, \frac{1}{a})]$, where $[(b, a)] + [(-\frac{b}{a^2}, \frac{1}{a})] = [(0, 1)]$.
Hence $(F, +)$ is an abelian group. Further, given $[(b, a)], [(d, c)], [(f, e)]$ in $F$,

\begin{equation*}
    [(b, a)]\times([(d, c)]\times[(f, e)])
        = [(b, a)]\times[(df, ce)]
        = [(bdf, ace)]
        = [(bd, ac)]\times[(f, e)]
        = ([(b, a)]\times[(d, c)])\times[(f, e)]
\end{equation*}

so multiplication is associative. For $(F, +, \times)$ to form a commutative ring, it remains
to show that the distributive laws hold. Firstly,

\begin{align*}
    [(b, a)]\times([(d, c)] + [(f, e)]) &= [(b, a)]\times[(cf + de, ce)]\\
    &= [(bcf + bde, ace)]\\
    &= [(abcf + abde, a^2ce)]\\
    &= [(bd, ac)] + [(bf, ae)]\\
    &= [(b, a)]\times[(d, c)] + [(b, a)]\times[(f, e)]
\end{align*}

Secondly,

\begin{align*}
    ([(b, a)] + [(d, c)])\times[(f, e)] &= [(ad + bc, ac)]\times[(f, e)]\\
    &= [(adf + bcf, ace)]\\
    &= [(adef + bcef, ace^2)]\\
    &= [(bf, ae)] + [(df, ce)]\\
    &= [(b, a)]\times[(f, e)] + [(d, c)]\times[(f, e)]
\end{align*}

so $(F, +, \times)$ forms a commutative ring. Now, to show that $F$ is a field, it remains
to show that $F$ is a commutative division ring. That is, given $[(b, a)] \neq [(0, 1)]$ in $F$,
there exists a multiplicative inverse for that element. The multiplicative identity is
$[(1, 1)]$ as $[(b, a)]\times([(1, 1)] = [(b, a)]$. $[(b^{-1}, a^{-1}]$ is the multiplicative
inverse element. This always exists for non-zero elements of $F$, as $a \neq 0$ ever, and if
$b = 0$ then $(b, a) \sim (0, 1)$, which contradicts $[(b, a)]$ being non-zero. Therefore,
$F$ is a commutative division ring, and is thus a field.

\end{document}
