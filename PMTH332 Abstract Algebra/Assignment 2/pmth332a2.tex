\documentclass{article}

\usepackage{amsmath}
\usepackage{amssymb}
\usepackage{listings}
\usepackage{color}
\usepackage{geometry}
\geometry{
    a4paper,
    left=25mm,
    top=25mm
}

\definecolor{codegreen}{rgb}{0,0.6,0}
\definecolor{codegray}{rgb}{0.5,0.5,0.5}
\definecolor{codepurple}{rgb}{0.58,0,0.82}
\definecolor{backcolour}{rgb}{0.95,0.95,0.92}
    
\lstdefinestyle{mystyle}{
    backgroundcolor=\color{backcolour},   
    commentstyle=\color{codegreen},
    keywordstyle=\color{magenta},
    numberstyle=\tiny\color{codegray},
    stringstyle=\color{codepurple},
    basicstyle=\footnotesize,
    breakatwhitespace=false,         
    breaklines=true,                 
    captionpos=b,                    
    keepspaces=true,                 
    numbers=left,                    
    numbersep=5pt,                  
    showspaces=false,                
    showstringspaces=false,
    showtabs=false,                  
    tabsize=4
}
\lstset{style=mystyle}

\newcommand\tab[1][1cm]{\hspace*{#1}}

\setlength{\parindent}{0pt}

\title{PMTH332 Assignment 2}
\date{28 July 2018}
\author{Jayden Turner (SN 220188234)}

\begin{document}
\maketitle
\pagenumbering{arabic}

\section*{Question 1}

Let $\phi := \ln:\mathbb{R}^+ \rightarrow \mathbb{R}, x \mapsto \ln x$. For $x, y \in H$, $\ln(xy) = \ln x + \ln y$, so $\phi$ is a homomorphism.
Now let $\psi := \text{exp}:\mathbb{R} \rightarrow \mathbb{R}^+, x \mapsto e^x$. Then $e^{x + y} = e^x e^y$, so $\psi$ is a homomorphism. Further,
$\psi = \phi^{-1}$, so $\phi$ is an isomorphism. Therefore, $G \cong H$.

\section*{Question 2}

Let $\phi: G \rightarrow H$ and $\psi: H \rightarrow K$ be isomorphisms. Then there exists inverses $\phi^{-1}: H \rightarrow G$ and $\psi^{-1}: K \rightarrow H$
of $\phi$ and $\psi$. By Proposition 3.6, $\psi \circ \phi$ is a homomorphism. Observe that

\begin{align*}
    (\phi^{-1} \circ \psi^{-1}) \circ (\psi \circ \phi) &= \phi^{-1} \circ (\psi^{-1} \circ (\psi \circ \phi))\\
    &= \phi^{-1} \circ((\psi^{-1} \circ \psi) \circ \phi)\\
    &= \phi^{-1} \circ(id_H \circ \phi)\\
    &= \phi^{-1} \circ \phi\\
    &= id_G
\end{align*}

\begin{align*}
    (\psi \circ \phi) \circ (\phi^{-1} \circ \psi^{-1}) &= (\psi \circ (\phi \circ (\phi^{-1} \circ \psi^{-1}))\\
    &= \psi \circ ((\phi \circ \phi^{-1}) \circ \psi^{-1})\\
    &= \psi \circ (id_G \circ \psi^{-1}\\
    &= \psi \circ \psi^{-1}\\
    &= id_K
\end{align*}

Therefore $\psi \circ \phi: G \rightarrow K$ has inverse $\phi^{-1} \circ \psi^{-1}: K \rightarrow G$ and is an isomorphism.

\section*{Question 3}

Let $H \subseteq G$ and define $x \sim_H y : \leftrightarrow xy^{-1} \in H$.

\hfill \break
Suppose $H \leq G$. Then if $x \in H$, $e = xx^{-1} \in H$. Therefore, $x \sim_H x$.

\hfill \break
As $H$ is a subgroup, for $x \in y$, $xy^{-1} \in H$. By the existence of inverses in groups, we thus have $(xy^{-1})^{-1} = yx^{-1} \in H$. Therefore, if
$x \sim_H y$ then $y \sim_H x$.

\hfill \break
If $x, y, z \in H$ we have $xy^{-1} \in H$ and $yz^{-1} \in H$. By closure of multiplication in groups, we then have
$(xy^{-1})(yz^{-1}) = xz^{-1} \in H$. Therefore $\sim_H$ is transitive and thus satisfies all criteria of an equivalence relation

\hfill \break
Conversely, suppose that $H \subseteq G$ and $x \sim_H y : \leftrightarrow xy^{-1} \in H$ defines an equivalence relation. Consider for $x \in H$ the equivalence class
$[x] = \{y \in H | xy^{-1} \in H\}$. By reflexivity of $\sim_H$, we have $x \in [x] \implies xx^{-1} = e \in H$. Let $h \in H$ and suppose that $h \notin [x]$. Then
$xh^{-1} \notin H$. However, as equivalence relations partition the set they act upon, there is some $y \in H$ such that $h \in [y] \implies yh^{-1} \in H$. Therefore,
given $x, y \in H$ we must have that $xy^{-1} \in H$ and thus $H$ is a subgroup of $G$.

\section*{Question 4}

a) Let $\phi, \psi \in \text{Aut}(G)$. Then both $\phi^{-1}: G \rightarrow G$ and $\psi^{-1}: G \rightarrow G$ exist and are in $\text{Aut}(G)$. Further,
$\upsilon := \phi \circ \psi^{-1}: G \rightarrow G$ is an isomorphism by the result of Question 2, and is therefore in Aut$(G)$. Therefore, by Proposition 4.5, 
Aut$(G) \leq S(G)$.

\hfill \break
b) Let $g \in G$ and define $\phi_g: G \rightarrow G, x \mapsto gxg^{-1}$. As $G$ is a group, it is closed under multiplication, so the function is
well defined. Consider $x, y \in G$. Then,

\begin{equation*}
    \phi_g(xy) = gxyg^{-1} = gxg^{-1}gyg^{-1} =\phi_g(y)\phi_g(y)
\end{equation*}

so $\phi_g$ is a homomorphism. Now let $x, x' \in G$ such that $\phi_g(x) = \phi_g(x')$. Then,

\begin{align*}
    gxg^{-1} &= gx'g^{-1}\\
    gxg^{-1}g &= gx'g^{-1}g\\
    g^{-1}xe &= g^{-1}gx'e\\
    ex &= ex'\\
    \implies x &= x'
\end{align*}

So $\phi_g$ is injective. Now let $y \in G$. Define $x = g^{-1}yg$. Then

\begin{equation*}
    \phi_g(x) = \phi_g(g^{-1}yg) = gg^{-1}yg^{-1}g = eye = y
\end{equation*}

So $\phi_g$ is surjective and thus bijective. Therefore, $\phi_g$ is an isomorphism from $G$ to $G$ and is thus an automorphism.

\hfill \break
c) Let $x, y \in G$. Then Inn$(xy) = \phi_{xy}$. Let $g \in G$. Then,

\begin{equation*}
    \phi_{xy}(g) = xyg(xy)^{-1} = xygy^{-1}x^{-1} = x(ygy^{-1})x^{-1} = \phi_x(\phi_y(g)) = (\phi_x \circ \phi_y)(g)
\end{equation*}

Therefore, Inn$(xy) = \phi_{xy} = \phi_x \circ \phi_y$, so Inn is a homomorphism.

\end{document}