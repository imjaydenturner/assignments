\documentclass{article}

\usepackage{amsmath}
\usepackage{amssymb}
\usepackage{listings}
\usepackage{color}
\usepackage{hhline}
\usepackage{geometry}
\geometry{
    a4paper,
    left=25mm,
    top=25mm
}

\definecolor{codegreen}{rgb}{0,0.6,0}
\definecolor{codegray}{rgb}{0.5,0.5,0.5}
\definecolor{codepurple}{rgb}{0.58,0,0.82}
\definecolor{backcolour}{rgb}{0.95,0.95,0.92}
    
\lstdefinestyle{mystyle}{
    backgroundcolor=\color{backcolour},   
    commentstyle=\color{codegreen},
    keywordstyle=\color{magenta},
    numberstyle=\tiny\color{codegray},
    stringstyle=\color{codepurple},
    basicstyle=\footnotesize,
    breakatwhitespace=false,         
    breaklines=true,                 
    captionpos=b,                    
    keepspaces=true,                 
    numbers=left,                    
    numbersep=5pt,                  
    showspaces=false,                
    showstringspaces=false,
    showtabs=false,
    tabsize=4
}
\lstset{style=mystyle}

\newcommand\tab[1][1cm]{\hspace*{#1}}
\setlength{\parindent}{0pt}

\iffalse <Subject> Assignment <Assignment number> \fi
\title{PMTH332 Assignment 4}
\iffalse <dd> <Month> <yyyy> \fi
\date{25 August 2018}
\author{Jayden Turner (SN 220188234)}

\begin{document}
\maketitle
\pagenumbering{arabic}

\section*{Question 1}

Let $M$ and $N$ be normal subgroups of $G$ such that
$M \trianglelefteq N \trianglelefteq G$. Consider $N/M \trianglelefteq G/M$. If $M$
is a maximal normal subgroup of $G$, then either

\begin{align*}
    N &= G & \text{or} && N &= M\\
    \implies N/M &= G/M & && \implies N/M &= M/M = \{M\}
\end{align*}

Therefore the only two normal subgroups of $G/M$ are $G/M$ and $\{M\}$, which is the
identity element. Therefore $G/M$ is simple.

\hfill \break
Suppose $M \trianglelefteq G$ and $G/M$ is simple. If
$M \trianglelefteq N \trianglelefteq G$, then $N/M \trianglelefteq G/M$. As $G/M$
is simple, either $N/M = G/M \implies N = G$ or $N/M = \{M\} \implies N = M$. Therefore
$M$ is a maximal normal subgroup of $G$.

\section*{Question 2}

Let $m = \text{ord}(a_1, ..., a_n)$. Then

\begin{equation} \label{eq:2-1}
    (a_1, ..., a_n)^m = (a_1^m, ..., a_n^m) = (e_1, ..., e_n)
\end{equation}

where $e_i$ is the identity element of group $G_i$. Therefore, for each $a_i$, there
exists an integer $q_i$ such that $m = q_i\text{ord}(a_i)$. As $m$ is the smallest
positive integer such that (\ref{eq:2-1}) holds, $m$ must be the least common
multiple of the orders of each $a_i$. That is,

\begin{equation*}
    \text{ord}(a_1, ..., a_n) = \text{lcm}(\text{ord}(a_1), ..., \text{ord}(a_n))
\end{equation*}

\section*{Question 3}

Consider the group

\begin{equation} \label{eq:3-1}
(\mathbb{Z}/m\mathbb{Z})\times(\mathbb{Z}/n\mathbb{Z})
    = \{(a, b) | a \in \mathbb{Z}/m\mathbb{Z}, b \in \mathbb{Z}/m\mathbb{Z}\}
\end{equation}

This group has $mn$ elements and is cyclic. By the result of Question 2,
the order of (\ref{eq:3-1}) is $d = \text{lcm}(m, n)$.

\hfill \break
By the classification of cyclic groups, this group is isomorphic to
$\mathbb{Z}/mn\mathbb{Z}$ if and only if the order $d = mn$. That is,
if and only if $d = \text{lcm}(m, n) = mn = \text{lcm}(m, n)\text{gcd}(m, n)$.
Therefore, $(\mathbb{Z}/m\mathbb{Z})\times(\mathbb{Z}/n\mathbb{Z})
\cong \mathbb{Z}/mn\mathbb{Z}$ if and only if $\text{gcd}(m, n) = 1$.

\newpage
\section*{Question 4}

The prime factorisation of $324$ is $324 = 2^2\cdot 3^4$. By the first Sylow theorem,
$G$ has subgroups of orders $2, 2^2 = 4, 3, 3^2 = 9, 3^3 = 27$ and $3^4 = 81$. Let $H$
be a subgroup of $G$ of order $10$. Then there is an element $g \in G$
of order $10$. By Corollary 6.17, it must hold that $\text{ord}(g)||G|$. However,
$10 \nmid 324$, so this is a contradiction, and $G$ has no subgroups of order $10$.
\end{document}