\documentclass{article}

\usepackage{amsmath}
\usepackage{amssymb}
\usepackage{listings}
\usepackage{color}
\usepackage{geometry}
\geometry{
    a4paper,
    left=25mm,
    top=25mm
}

\definecolor{codegreen}{rgb}{0,0.6,0}
\definecolor{codegray}{rgb}{0.5,0.5,0.5}
\definecolor{codepurple}{rgb}{0.58,0,0.82}
\definecolor{backcolour}{rgb}{0.95,0.95,0.92}
    
\lstdefinestyle{mystyle}{
    backgroundcolor=\color{backcolour},   
    commentstyle=\color{codegreen},
    keywordstyle=\color{magenta},
    numberstyle=\tiny\color{codegray},
    stringstyle=\color{codepurple},
    basicstyle=\footnotesize,
    breakatwhitespace=false,         
    breaklines=true,                 
    captionpos=b,                    
    keepspaces=true,                 
    numbers=left,                    
    numbersep=5pt,                  
    showspaces=false,                
    showstringspaces=false,
    showtabs=false,
    tabsize=4
}
\lstset{style=mystyle}

\newcommand\tab[1][1cm]{\hspace*{#1}}

\setlength{\parindent}{0pt}

\title{PMTH339 Assignment 3}
\date{3 August 2018}
\author{Jayden Turner (SN 220188234)}

\begin{document}
\maketitle
\pagenumbering{arabic}

\section*{Question 1}

\begin{equation} \label{eq:1-1}
    y'' = 2
\end{equation}

\hfill \break
By inspection of the homogeneous DE $y'' = 0$, $y_1 = 1$ and $y_2 = x$ are a fundamental pair of solutions with Wronskian
$W = y_1y_2' - y_1'y_2 = 1$. By Theorem 6.1, the unique solution $y$ to (\ref{eq:1-1}) satisfying $y(0) = y'(0) = 0$ is given by

\begin{align*}
    y(x) &= y_2(x)\int_0^x \frac{y_1(t)}{W(t)}g(t) dt - y_1(x)\int_0^x \frac{y_2(t)}{W(t)}g(t) dt\\
    &= x\int_0^x 2 dt - \int_0^x 2t dt\\
    &= 2x^2 - x^2\\
    &= x^2
\end{align*}

\section*{Question 2}

\begin{equation} \label{eq:2-1}
    y'' + n^2y = \cos mx
\end{equation}

\hfill \break
The homogeneous DE $y'' + n^2y = 0$ has characteristic equation $\lambda^2 + n^2 = 0$, with solutions $\lambda = \pm in$. Thus, the homogeneous DE has solutions
$y_1 = A\cos nx$ and $y_2 = B\sin nx$.

\hfill \break
Assume solution to (\ref{eq:2-1}) is of the form $y = C_1\cos mx + C_2\sin mx$. Differentiating, we get $y' = -C_1\sin mx + C_2\cos mx$ and
$y'' = -C_1\cos mx - C_2\sin mx$. For this to be a solution, we require

\begin{align*}
    \cos mx &= y'' + n^2y\\
    &= -C_1\cos mx - C_2\sin mx + n^2(C_1\cos mx + C_2\sin mx)\\
    &= C_1(n^2 - 1)\cos mx + C_2(n^2 - 1)\sin mx
\end{align*}

Equating coefficients, we thus require

\begin{align*}
    C_1(n^2 - 1) &= 1 \implies C_1 = \frac{1}{n^2 - 1}\\
    C_2(n^2 - 1) &= 0 \implies C_2 = 0
\end{align*}

where $n \neq \pm 1$. Therefore the general solution to (\ref{eq:2-1}) is

\begin{equation} \label{eq:2-2}
    y(x) = A\cos nx + B\sin nx + \frac{1}{n^2 - 1}\cos mx
\end{equation}

$A$ and $B$ can be found according to the initial conditions $y(0) = 1$ and $y'(0) = 0$:

\begin{align*}
    y(0) &= 1 \implies 1 = A + \frac{1}{n^2 - 1} \implies A = \frac{n^2 - 2}{n^2 - 1}\\
    y'(0) &= 0 \implies 0 = B
\end{align*}

Therefore the particular solution according to the above initial conditions is

\begin{equation*}
    y(x) = \frac{1}{n^2 - 1}(\cos mx + (n^2 - 2)\cos nx)
\end{equation*}

\section*{Question 3}

\begin{equation} \label{eq:3-1}
    (2x + 1)y'' + (4x - 2)y' - 8y = 0
\end{equation}

$y_1 = e^{-2x} \implies y_1' = -2e^{-2x} \implies y_1'' = 4e^{-2x}$. Substituting these into (\ref{eq:3-1}),

\begin{align*}
    (2x + 1)y_1'' + (4x - 2)y_1' - 8y_1 &= (2x + 1)(4e^{-2x}) + (4x - 2)(-2e^{-2x}) - 8e^{-2x}\\
    &= 8xe^{-2x} + 4e^{-2x} - 8xe^{-2x} + 4e^{-2x} - 8e^{-2x}\\
    &= 0
\end{align*}

so $y_1$ is indeed a solution of (\ref{eq:3-1}). Dividing by the coefficient of $y''$, (\ref{eq:3-1}) becomes

\begin{equation} \label{eq:3-2}
    y'' + \frac{4x - 2}{2x + 1}y' - \frac{8}{2x + 1}y = 0
\end{equation}

as long as $x \neq -\frac{1}{2}$. Given $y_1$, Theorem 5.1 guarantees a linearly independent second solution $y_2$ given by

\begin{equation} \label{eq:3-3}
    y_2(x) = y_1(x)\int \frac{W(x)}{y_1(x)^2} dx
\end{equation}

where $W(x)$ is the Wronskian, calculated as follows:

\begin{align*}
    W(x) &= e^{-\int p(x) dx}\\
    &= e^{-\int \frac{4x - 2}{2x + 1} dx}\\
\end{align*}

Let $u = 2x + 1$. Then $x = \frac{u - 1}{2}$ and $\frac{du}{dx} = 2$. Then,

\begin{align*}
    W(x) &= e^{-\int \frac{u - 2}{u}du}\\
    &= e^{2\int \frac{1}{u}du}e^{-\int 1 du}\\
    &= e^{2\ln u}e^{-u}\\
    &= u^2e^{-u}\\
    &= (2x + 1)^2e^{-2x - 1}
\end{align*}

Using (\ref{eq:3-3}), a second solution to (\ref{eq:3-1}) is

\begin{align*}
    y_2(x) &= y_1(x)\int\frac{W(x)}{y_1(x)^2} dx\\
    &= e^{-2x}\int \frac{(2x + 1)^2e^{-2x - 1}}{e^{-4x}} dx\\
    &= e^{-2x}\int (2x + 1)^2e^{2x - 1} dx\\
    &= e^{-2x - 2}\int(2x + 1)^2e^{2x + 1} dx
\end{align*}

Let $u = 2x + 1$, then the integral becomes

\begin{equation*}
    y_2(x) = \frac{e^{-2x - 2}}{2} \int u^2e^u du\\
\end{equation*}

Set $v = u^2, w' = e^u$ and integrate by parts to get $\int u^2e^u du = u^2e^u - 2\int ue^u du$. Integrating by parts again, setting $v = u, w' = e^u$, this
becomes $\int u^2e^u = u^2e^u - 2ue^u + 2e^u$. Therefore,

\begin{align*}
    y_2(x) &= e^{-2x - 2}(u^2e^u - 2ue^u + 2e^u)\\
    &= e^{-2x - 2}(e^{2x + 1})((2x + 1)^2 - 2(2x + 1) + 2)\\
    &= \frac{4x^2 + 1}{2e}
\end{align*}

\section*{Question 4}

\begin{equation} \label{eq:4-1}
    (4x^2 - x)y'' + 2(2x - 1)y' - 4y = 0
\end{equation}

\hfill \break
\begin{equation} \label{eq:4-2}
    (4x^2 - x)y'' + 2(2x - 1)y' - 4y = 12x^2 - 6x
\end{equation}

$y_1 = \frac{1}{x} \implies y_1' = -\frac{1}{x^2} \implies y_1'' = \frac{2}{x^3}$. Therefore, substituting $y_1$ into (\ref{eq:4-1}),

\begin{align*}
    (4x^2 - x)y_1'' + 2(2x - 1)y_1' - 4y_1 &= (4x^2 - x)\frac{2}{x^3} + 2(2x - 1)\left(-\frac{1}{x^2}\right) - \frac{4}{x}\\
    &= \frac{8}{x} - \frac{2}{x^2} - \frac{4}{x} + \frac{2}{x^2} - \frac{4}{x}\\
    &= 0
\end{align*}

so $y_1$ indeed solves (\ref{eq:4-1}). Rearrange (\ref{eq:4-1}) and (\ref{eq:4-2})to get

\begin{equation} \label{eq:4-3}
    y'' + \frac{2(2x - 1)}{x(4x - 1)}y' - \frac{4}{x(4x - 1)} = 0
\end{equation}

\hfill \break
\begin{equation} \label{eq:4-4}
    y'' + \frac{2(2x - 1)}{x(4x - 1)}y' - \frac{4}{x(4x - 1)} = \frac{12x - 6}{4x - 1}
\end{equation}

where $x \neq 0, \frac{1}{4}$. Similiarly to Question 3, Theorem 5.1 guarantees a second solution by (\ref{eq:3-3}) using the Wronskian

\begin{align*}
    W &= e^{-2\int\frac{2x - 1}{x(4x - 1)} dx}\\
    &= e^{-2\left(\int \frac{1}{x} - \frac{2}{4x - 1}dx\right)}\\
    &= e^{-2\int \frac{1}{x} dx + 5\int \frac{1}{4x - 1} dx}\\
    &= e^{-2\ln x}e^{\ln(4x - 1)}\\
    &= \frac{4x - 1}{x^2}
\end{align*}

which exists and is non-zero when $x \neq 0, \frac{1}{4}$. This second solution $y_2$ is then

\begin{align*}
    y_2 &= y_1\int \frac{W(x)}{y_1(x)^2} dx\\
    &= \frac{1}{x} \int \frac{4x - 1}{x^2} x^2 dx\\
    &= \frac{1}{x} \int 4x - 1 dx\\
    &= \frac{1}{x}(2x^2 - x)\\
    &= 2x - 1
\end{align*}

Now, given linearly independent $y_1$ and $y_2$, Theorem 6.1 says that the unique solution $y$ to (\ref{eq:4-4}) that satisfies initial conditions
$y(x_0) = y'(x_0) = 0$ is given by

\begin{equation} \label{eq:4-5}
    y(x) = y_2(x) \int_{x_0}^x \frac{y_1(t)}{W(t)}g(t) dt - y_1(x) \int_{x_0}^x \frac{y_2(t)}{W(t)} g(t) dt
\end{equation}

$x_0$ is an arbitrary point in the domain. In this case, choose $x_0 = 1$. Then (\ref{eq:4-5}) evalutates as

\begin{align*}
    y(x) &= (2x - 1)\int_1^x \frac{1}{t}\frac{t^2}{4t - 1}{6(2t - 1)}{4t - 1} dt
        - \frac{1}{x} \int_1^x (2t - 1)\frac{t^2}{4t - 1}\frac{6(2t - 1)}{4t - 1}dt\\
    &= 6(2x - 1)\int_1^x \frac{t(2t - 1)}{(4t - 1)^2} dt - \frac{6}{x} \int_1^x\frac{t^2(2t - 1)^2}{(4t - 1)^2} dt\\
    &= 6(2x - 1)\int_1^x \frac{1}{8} - \frac{1}{8(4t - 1)^2} dt
        - \frac{6}{x} \int_1^x \frac{1}{4}t^2 - \frac{1}{8}t - \frac{1}{64} + \frac{1}{64(4t - 1)^2} dt\\
    &= \frac{3(2x - 1)}{4} \left[t + \frac{1}{4(4t - 1}\right]_1^x
        - \frac{6}{x}\left[\frac{1}{12}t^3 - \frac{1}{16}t^2 - \frac{1}{64}t - \frac{1}{256(4t - 1)}\right]_1^x\\
    &= \frac{6x - 3}{4} \left(x + \frac{1}{4(4x - 1)} - \frac{1}{12}\right)
        - \frac{6}{x}\left(\frac{1}{12}x^3 - \frac{1}{16}x^2 - \frac{1}{64}x - \frac{1}{256(4x - 1)} + \frac{1}{768}\right)\\
    &= \frac{3}{2}x^2 + \frac{3x}{8(4x - 1)} - \frac{3}{24}x - \frac{3}{4}x - \frac{3}{16(4x - 1)} + \frac{3}{48}
        - \frac{1}{2}x^2 + \frac{3}{8}x + \frac{3}{32} + \frac{3}{128x(4x - 1)} - \frac{1}{128x}\\
    &= x^2 - \frac{1}{2}x + \frac{1}{8} + \frac{6x - 3}{16(4x - 1)} + \frac{3}{128x(4x - 1)} - \frac{1}{128x}\\
    &= x^2 - \frac{1}{2}x + \frac{1}{8} + \frac{8x(6x - 3) + 3 - (4x - 1)}{128x(4x - 1)}\\
    &= x^2 - \frac{1}{2}x + \frac{1}{8} + \frac{48x^2 - 24x + 3 - 4x + 1}{128x(4x - 1)}\\
    &= x^2 - \frac{1}{2}x + \frac{1}{8} + \frac{4(12x^2 - 7x + 1)}{128x(4x - 1)}\\
    &= x^2 - \frac{1}{2}x + \frac{1}{8} + \frac{(3x - 1)(4x - 1)}{32x(4x - 1)}\\
    &= x^2 - \frac{1}{2}x + \frac{1}{8} + \frac{3x - 1}{32x}\\
    &= x^2 - \frac{1}{2}x + \frac{7}{32} - \frac{1}{32x}
\end{align*}

Therefore, the general solution to (\ref{eq:4-2}) is

\begin{equation*}
    y(x) = \frac{A}{x} + B(2x - 1) + x^2 - \frac{1}{2}x + \frac{7}{32} - \frac{1}{32x}
\end{equation*}

where $A, B$ are constants.

\section*{Question 5}

\begin{equation} \label{eq:5-1}
    y'' = \frac{y'}{x} + \frac{x^2}{y'}
\end{equation}

Multiply both sides by $y'$ to get

\begin{equation} \label{eq:5-2}
    y'y'' = \frac{(y')^2}{x} + x^2
\end{equation}

Let $\phi = (y')^2$. Then $\phi' = 2y'y''$. Therefore, (\ref{eq:5-2}) is transformed into the first order DE

\begin{equation} \label{eq:5-3}
    \phi' - \frac{2}{x}\phi = 2x^2
\end{equation}

Let $I(x) = e^{\int -\frac{2}{x} dx} = e^{-2\ln x} = \frac{1}{x^2}$. Multiply (\ref{eq:5-3}) by $I$ to get

\begin{equation} \label{eq:5-4}
    \frac{1}{x^2}\phi' - \frac{2}{x^3}\phi = 2
\end{equation}

By the chain rule, the left hand side is equal to $\frac{d}{dx}\left(\frac{1}{x^2}\phi\right)$. Integrate both sides with respect
to x to get

\begin{align}
    \frac{1}{x^2}\phi &= 2x + C\nonumber\\
    \phi &= 2x^3 + Cx^2\nonumber\\
    (y')^2 &= 2x^3 + Cx^2\nonumber\\
    y' &= \sqrt{2x^3 + Cx^2} \label{eq:5-5}
\end{align}

Using the intial condition $y'(2) = 4$ yields $C = 0$. Substituting this value and integrating both sides with respect to x
again,

\begin{align}
    y' &= \sqrt{2x^3}\nonumber\\
    &= \sqrt{2}x^{\frac{3}{2}}\nonumber\\
    y(x) &= \frac{2\sqrt{2}}{5}x^{\frac{5}{2}} + D \label{eq:5-6}
\end{align}

Using the initial condition $y'(2) = 0$ yields $D = -\frac{16}{5}$. Therefore, the solution to (\ref{eq:5-1}) satisfying the given
initial conditions is

\begin{equation*}
    y(x) = \frac{2\sqrt{2}}{5}x^\frac{5}{2} - \frac{16}{5}
\end{equation*}

\end{document}