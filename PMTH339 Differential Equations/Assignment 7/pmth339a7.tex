\documentclass{article}

\usepackage{amsmath}
\usepackage{amssymb}
\usepackage{listings}
\usepackage{color}
\usepackage{hhline}
\usepackage{geometry}
\geometry{
    a4paper,
    left=25mm,
    top=25mm
}

\definecolor{codegreen}{rgb}{0,0.6,0}
\definecolor{codegray}{rgb}{0.5,0.5,0.5}
\definecolor{codepurple}{rgb}{0.58,0,0.82}
\definecolor{backcolour}{rgb}{0.95,0.95,0.92}
    
\lstdefinestyle{mystyle}{
    backgroundcolor=\color{backcolour},   
    commentstyle=\color{codegreen},
    keywordstyle=\color{magenta},
    numberstyle=\tiny\color{codegray},
    stringstyle=\color{codepurple},
    basicstyle=\footnotesize,
    breakatwhitespace=false,         
    breaklines=true,                 
    captionpos=b,                    
    keepspaces=true,                 
    numbers=left,                    
    numbersep=5pt,                  
    showspaces=false,                
    showstringspaces=false,
    showtabs=false,
    tabsize=4
}
\lstset{style=mystyle}

\newcommand\tab[1][1cm]{\hspace*{#1}}
\setlength{\parindent}{0pt}

\iffalse <Subject> Assignment <Assignment number> \fi
\title{PMTH339 Assignment 7}
\iffalse <dd> <Month> <yyyy> \fi
\date{14 September 2018}
\author{Jayden Turner (SN 220188234)}

\begin{document}
\maketitle
\pagenumbering{arabic}

\section*{Question 1}

\begin{equation} \label{eq:1-1}
    y'' + (3 - \sin x)y = 0
\end{equation}

Let $a(x) = 3 - \sin x$. Then for all $x$, $a(x) \geq 2$. Therefore,
if $u$ is a solution to (\ref{eq:1-1}), Corollary 14.3 states that $u$ has an
increasing sequence of zeroes $\{\alpha_i\}$ where $0 < \alpha_1 \leq \frac{\pi}{\sqrt{2}}$
and $\alpha_{i + 1} - \alpha_i \leq \frac{\pi}{\sqrt{2}}$ for all $i$.

\hfill \break
Given the two inequalities above, we can iteratively determine an upper bound for the
first $n$ zeroes. Firstly, $0 < \alpha_1 \leq \frac{\pi}{\sqrt{2}}$. Then, as
$\alpha_2 - \alpha_1 \leq \frac{\pi}{\sqrt{2}}$ it must hold that

\begin{equation*}
    \alpha_2 = \alpha_2 - \alpha_1 + \alpha_1 \leq 
        \frac{\pi}{\sqrt{2}} + \frac{\pi}{\sqrt{2}} = \sqrt{2}\pi
\end{equation*}

Repeating the process we get that $\alpha_3 \leq \frac{3\pi}{\sqrt{2}}$. Therefore, as
$\sqrt{2}\pi \leq 2\pi \leq \frac{3\pi}{\sqrt{2}}$, there are either 2 or 3 zeroes in the
range $0 \leq x \leq 2\pi$.

\section*{Question 2}

\section*{Question 3}

\section*{Question 4}

\end{document}