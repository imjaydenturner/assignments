\documentclass{article}

\usepackage{amsmath}
\usepackage{amssymb}
\usepackage{bm}
\usepackage{listings}
\usepackage{color}
\usepackage{hhline}
\usepackage{geometry}
\geometry{
    a4paper,
    left=25mm,
    top=25mm
}

\definecolor{codegreen}{rgb}{0,0.6,0}
\definecolor{codegray}{rgb}{0.5,0.5,0.5}
\definecolor{codepurple}{rgb}{0.58,0,0.82}
\definecolor{backcolour}{rgb}{0.95,0.95,0.92}
    
\lstdefinestyle{mystyle}{
    backgroundcolor=\color{backcolour},   
    commentstyle=\color{codegreen},
    keywordstyle=\color{magenta},
    numberstyle=\tiny\color{codegray},
    stringstyle=\color{codepurple},
    basicstyle=\footnotesize,
    breakatwhitespace=false,         
    breaklines=true,                 
    captionpos=b,                    
    keepspaces=true,                 
    numbers=left,                    
    numbersep=5pt,                  
    showspaces=false,                
    showstringspaces=false,
    showtabs=false,
    tabsize=4
}
\lstset{style=mystyle}

\newcommand\tab[1][1cm]{\hspace*{#1}}
\setlength{\parindent}{0pt}

\iffalse <Subject> Assignment <Assignment number> \fi
\title{PMTH339 Assignment 8}
\iffalse <dd> <Month> <yyyy> \fi
\date{28 September 2018}
\author{Jayden Turner (SN 220188234)}

\begin{document}
\maketitle
\pagenumbering{arabic}

\section*{Question 1}

\begin{equation} \label{eq:1-1}
    (1 - x^2)y'' - xy' + \alpha^2y = 0
\end{equation}

Divide through by $(1 - x^2)$ to get

\begin{equation} \label{eq:1-2}
    y'' - \frac{x}{1 - x^2}y' + \frac{\alpha^2}{1 - x^2}y = 0
\end{equation}

Multiply by $I(x)$, where $I(x)$ is

\begin{equation} \label{eq:1-3}
    I(x) = e^{\int-\frac{x}{1 - x^2} dx}
        = e^{\frac{1}{2}\ln|1 - x^2|}
        = \sqrt{1 - x^2}
\end{equation}

Therefore

\begin{align}
    I(x)(\ref{eq:1-2}) \implies
    0 &= \sqrt{1 - x^2}y'' - \frac{x}{\sqrt{1 - x^2}}y' + \frac{\alpha^2}{\sqrt{1 - x^2}}y\nonumber\\
    &= (\sqrt{1 - x^2}y')' + \frac{\alpha^2}{\sqrt{1 - x^2}}y \label{eq:1-4}
\end{align}

(\ref{eq:1-4}) is in the form $(p(x)y')' + q(x)y = 0$, with $p(x) = \sqrt{1 - x^2}$ and
$q(x) = \frac{\alpha^2}{\sqrt{1 - x^2}}$ defined and with continuous derivatives on the interval
$(-1, 1)$.

\hfill\break
The Chebyshev polynomials $T_n(x)$ and $U_n(x)$ hold certain properties that are useful here.
Firstly, $T_n(x)$ and $U_n(x)$ solve (\ref{eq:1-1}) and thus (\ref{eq:1-4}) for $\alpha = n$.
Secondly, the derivatives of $T_n(x)$ can be defined in terms of $U_{n - 1}(x)$ as
$T_n'(x) = nU_{n - 1}(x)$. Finally, they hold the following values at $x = -1, 1$:

\begin{align} \label{eq:1-5}
    T_n(-1) = (-1)^n && U_n(-1) = (n + 1)(-1)^n && T_n(1) = 1 && U_n(1) = n + 1
\end{align}

From this, we can see that the Chebyshev polynomials $T_n$ satisfy (\ref{eq:1-4}) and the
following boundary conditions

\begin{align} \label{eq:1-6}
    T_n(-1) - \frac{1}{n^2}T_n'(-1) = 0 && T_n(1) + \frac{1}{n^2}T_n'(1) = 0
\end{align}

Therefore, the Chebyshev polynomials $T_n$ are eigenfunctions of the linear operator
$L[y] = (-py')'$ corresponding to non-negative integer $n$ eigenvalues.

\hfill\break
Let $n$ and $m$ be distinct non-negative integers. Then, as $T_n$ and $T_m$
are eigenfunctions corresponding to distinct eigenvalues, Theorem 19.1 implies
that they are $r$-orthogonal for any function $r$. That is, the inner product
$\langle rT_n, T_m \rangle = 0$. In particular, if $r(x) = \frac{1}{\sqrt{1 - x^2}}$
we get that

\begin{equation*}
    \langle rT_n, T_m \rangle = \int_{-1}^{1}(1 - x^2)^{-1}T_n(x)T_m(x) dx = 0
\end{equation*}

as required.

\section*{Question 2}

\begin{align} 
    u'' + \lambda u &= 0 \label{eq:2-1a}\\
    u'(0) = u'(1) &= 0 \label{eq:2-1b}
\end{align}

We consider three cases for $\lambda$.

\hfill\break
\underline{$\bm{\lambda > 0}$:}

If $\lambda > 0$ then (\ref{eq:2-1a}) has solution
$u(x) = A\cos\sqrt\lambda x + B\sin\sqrt\lambda x$. The boundary condition
$u'(0) = 0$ implies that $-B\cos\sqrt\lambda 0 = -B = 0$, so $B$ must be 0.
The second boundary condition gives $A\sin\sqrt\lambda = 0$. This has a
non-trivial solution when $\sqrt\lambda = n\pi$, $n \in \mathbb{Z}^+$.
Therefore the system (\ref{eq:2-1a}), (\ref{eq:2-1b}) has eigenvalues
$\lambda_n = n^2\pi^2$ with corresponding eigenfunctions
$\phi_n(x) = \cos n\pi x$.

\hfill\break
\underline{$\bm{\lambda = 0}$:}

If $\lambda = 0$ then $(\ref{eq:2-1a})$ has solution
$u(x) = Ax + B$. The first boundary condition requires $A = 0$,
and $u(x) = B$ satisfies the second. Therefore the system has an eigenvalue
$\lambda_0 = 0$ with corresponding eigenfunction $\phi_0(x) = 1$.

\hfill\break
\underline{$\bm{\lambda < 0}$:}

If $\lambda < 0$, then the differential equation has solution
$u(x) = Ae^{\sqrt{-\lambda} x} + Be^{-\sqrt{-\lambda} x}$. The first boundary
condition gives $0 = A\sqrt{-\lambda}e^0 - B\sqrt{-\lambda}e^0$ which holds as
long as $A = B$. The second condition, gives
$0 = A\sqrt{\lambda}(e^{\sqrt{-\lambda}} - e^{-\sqrt{-\lambda}})$. However,
this is only true for $\lambda = 0$, and so the system has no negative
eigenvalues.

\hfill \break
Therefore, (\ref{eq:2-1a}), (\ref{eq:2-1b}) has eigenvalues and eigenfunctions
given by

\begin{equation*}
    \lambda_n = n^2\pi^2,\; \phi_n(x) = \cos n\pi x
\end{equation*}

for non-negative integer $n$.

\section*{Question 3}

\begin{align}
    u'' + ku &= F(x) \label{eq:3-1a}\\
    u'(0) = u'(1) &= 0 \label{eq:3-1b}
\end{align}

Two solutions to the homogenous differential equation are
$u_1(x) = \cos\sqrt k x$ and $u_2(x) = \sin\sqrt k x$, which have Wronskian
$W = \sqrt k$. The general solution of (\ref{eq:3-1a}) is therefore

\begin{align}
    u(x) &= \sin\sqrt k x \int_0^x \frac{\cos\sqrt k t}{\sqrt k} F(t) dt
        -\cos\sqrt k x \int_0^x \frac{\sin\sqrt k t}{\sqrt k} F(t)dt
        + A\cos\sqrt k x + B\sin\sqrt k x \nonumber\\
    &= \frac{1}{\sqrt k}\int_0^x F(t)(\sin\sqrt k x\cos\sqrt k t
        - \cos\sqrt k x \sin\sqrt k t) dt + A\cos\sqrt k x
        + B\sin\sqrt k x \nonumber\\
    &= \frac{1}{\sqrt{k}} \int_0^x \sin(\sqrt k(x - t))F(t) dt
        + A\cos\sqrt k x + B\sin\sqrt k x \label{eq:3-2}
\end{align}

Taking the first derivative, we get

\begin{equation} \label{eq:3-3}
    u'(x) = \frac{1}{\sqrt k}\int_0^x F(t)\cos(\sqrt k(x - t)) dt
        - A\sqrt k\sin\sqrt k x + B\sqrt k\cos\sqrt k x
\end{equation}

The first boundary requires $B = 0$. To satisfy the second boundary
condition, we need to choose $A$ so that

\begin{equation} \label{eq:3-4}
    \int_0^1 F(t)\cos(\sqrt k(x - t)) dt -kA\sin\sqrt k = 0
\end{equation}

\section*{Question 4}

\end{document}
