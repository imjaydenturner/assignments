\documentclass{article}

\usepackage{amsmath}
\usepackage{amssymb}
\usepackage{listings}
\usepackage{color}
\usepackage{hhline}
\usepackage{geometry}
\geometry{
    a4paper,
    left=25mm,
    top=25mm
}

\definecolor{codegreen}{rgb}{0,0.6,0}
\definecolor{codegray}{rgb}{0.5,0.5,0.5}
\definecolor{codepurple}{rgb}{0.58,0,0.82}
\definecolor{backcolour}{rgb}{0.95,0.95,0.92}
    
\lstdefinestyle{mystyle}{
    backgroundcolor=\color{backcolour},   
    commentstyle=\color{codegreen},
    keywordstyle=\color{magenta},
    numberstyle=\tiny\color{codegray},
    stringstyle=\color{codepurple},
    basicstyle=\footnotesize,
    breakatwhitespace=false,         
    breaklines=true,                 
    captionpos=b,                    
    keepspaces=true,                 
    numbers=left,                    
    numbersep=5pt,                  
    showspaces=false,                
    showstringspaces=false,
    showtabs=false,
    tabsize=4
}
\lstset{style=mystyle}

\newcommand\tab[1][1cm]{\hspace*{#1}}
\setlength{\parindent}{0pt}

\iffalse <Subject> Assignment <Assignment number> \fi
\title{PMTH339 Assignment 8}
\iffalse <dd> <Month> <yyyy> \fi
\date{21 September 2018}
\author{Jayden Turner (SN 220188234)}

\begin{document}
\maketitle
\pagenumbering{arabic}

\section*{Question 1}

\begin{align}
    (1 - x^2)y'' -& xy' + \alpha^2y = 0\nonumber\\
    y'' -& \frac{x}{1 - x^2}y' + \frac{\alpha^2}{1 - x^2}y = 0 \label{eq:1-1}
\end{align}

Define $I(x)$ by

\begin{equation} \label{eq:1-2}
    I(x) = e^{-\int \frac{x}{1 - x^2}}dx = e^{\frac{1}{2}\ln(1 - x^2)} = \sqrt{1 - x^2}
\end{equation}

Multiply (\ref{eq:1-1}) by (\ref{eq:1-2}) to get

\begin{align}
    \sqrt{1 - x^2}y'' -& \frac{x}{\sqrt{1 - x^2}}y' + \frac{\alpha^2}{\sqrt{1 - x^2}} = 0\nonumber\\
    &(-\sqrt{1 - x^2}y')' + \frac{\alpha^2}{\sqrt{1 - x^2}}y = 0 \label{eq:1-3}
\end{align}

This is in the form $L[y] = \lambda ry$, where $r(x) = \frac{\alpha^2}{\sqrt{1 - x^2}}$.
The Chebyshev polynomials $T_n$ solve (\ref{eq:1-3}) and satisfy the initial conditions

\begin{align} \label{eq:1-4}
    a_1T_n(-1) + a_2T_n'(-1) = 0 && b_1T_n(1) + b_2T_n'(1) = 0
\end{align}

Therefore, the Chebyshev polynomials are eigenfunctions of the system (\ref{eq:1-3}), (\ref{eq:1-4}),
and by Theorem 9.1, are $r$-orthogonal, where $r(x) = \frac{\alpha^2}{\sqrt{1 - x^2}}$. That is,

\begin{equation*}
    \int_{-1}^1 (1 - x^2)^{-1/2}T_m(x)T_n(x) dx = 0
\end{equation*}

for $m \neq n$.

\section*{Question 2}

\begin{align}
    u'' + \lambda u &= 0 \label{eq:2-1}\\
    u'(0) = u'(1) &= 0 \label{eq:2-2}
\end{align}

If $\lambda > 0$, then $u$ is of the form

\begin{equation*}
    u = A\cos\sqrt{\lambda}x + B\sin\sqrt{\lambda}x
\end{equation*}

Therefore $u' = -A\sqrt{\lambda}\sin\sqrt{\lambda}x + B\sqrt{\lambda}\cos\sqrt{\lambda}x$.
The first condition gives $B = 0$. Therefore $u' = -A\sqrt{\lambda}\sin\sqrt{\lambda}x$. The
second condition requires $-A\sqrt{\lambda}\sin\sqrt{\lambda} = 0$, which only holds for
$\lambda = n^2\pi^2$, where $n$ is an integer.

\hfill\break
If $\lambda = 0$, then $u = A$. For $\lambda < 0$, $u$ is of the form

\begin{equation*}
    u = Ae^{\sqrt{-\lambda}x} + Be^{-\sqrt{-\lambda}x}
\end{equation*}

Taking the first derivative and applying the first condition, we get $A = B$.
Making this substitution and applying the second condition, we get that

\begin{equation*}
    0 = \sqrt{-\lambda}A(e^{\sqrt{-\lambda}} - e^{-\sqrt{\lambda}})
\end{equation*}

Which only has solution when $\lambda = 0$, hence there are no solutions for $\lambda < 0$.

\hfill\break
Therefore, the eigenvalues of the system (\ref{eq:2-1}), (\ref{eq:2-2}) are
$\lambda_n = n^2\pi^2$ for integer $n \geq 0$, which correspond to eigenfunctions
$\phi_n = \cos\sqrt{\lambda_n}x$. Note that this includes the case for $\lambda = 0$, which
corresponds to eigenfunction $\phi_0 = \cos\sqrt{0}x = 1$.

\section*{Question 3}

\section*{Question 4}

\end{document}
